Mars surface is covered by sand and volcanic rocks. The first purpose of this project is to allow scientists to examine these rocks through the camera. To achieve it, it is needed to know their characteristics, especially their albedo. As a reminder, the albedo is the fraction of incident light which is reflected from a surface. We can assume that the power reflection of Martian rocks is the same than on Earth. Then, to cover a wild range of rocks, the albedos of a black and of a white stone would be considered. Charcoal, as a dark rock, is a powerful absorber of the sun radiation, with an albedo around 0.05. On the contrary, chalks are poor absorbers and their albedo reach 0.45 according to \cite{albedo}. In the following parts of the report, it will be taken for granted that Martian rocks have an albedo between 5 and 45\%.
