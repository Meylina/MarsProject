\label{climate}
Because of its distance to the Sun bigger than the one of Earth, Mars receives less solar energy \ref{Irradiance}. Additionally, as its atmosphere is thinner, there is only a negligible greenhouse effect, and hence the average temperature is around -63\textdegree C on the surface of Mars with wide variations between daylight and night. \\
Then, the obliquity of Mars is close to the one of Earth (respectively 25,19\textdegree and 23,44\textdegree) but the eccentricity of Mars is bigger (0,09332 against 0,01671 for Earth) which means that, if Mars has similar seasons to Earth, they have different intensity and duration during the martian year. Thus, the northern hemisphere has seasons less pronounced than the southern hemisphere because its aphelion is at the end of spring and its perihelion at the end of autumn. Thus, there are short and soft winters and long and fresh summers in the northern hemisphere. On the opposite, the southern hemisphere has very pronounced seasons with long and cold winters and short and warmer summers than in the northern hemisphere. That is why there are higher temperature differences in the south.