\subsection{Summary and Comparison with MER Cameras}
To validate the coherence of the characteristics of the designed camera, a comparative table \ref{tabCameras} with the MER cameras Navcam (Navigation Camera) and Pancam (Panoramic Camera) \cite{merengineeringcameras} can be found below.

\begin{table}[H]
\centering
\caption{Recap chart of the Designed Camera and Comparison with Navcam and Pancam (MER Cameras)}
\label{tabCameras}
\renewcommand{\arraystretch}{1.5}
\begin{tabular}{|l|c c c|}
\hline
Features & Designed Camera & Navcam & Pancam \\
\hline
CCD &  &  &  \\
Pixel size & $12 \times 12 $ microns & $12 \times 12 $ microns & $12 \times 12 $ microns \\ 
Resolution & $1024 \times 1024 $ & $1024 \times 1024 $ & $1024 \times 1024 $ \\ 
Spectral range & [400 - 800] nm & [600 - 800] nm & [400 - 1100] nm \\ 
Readout noise, 55$\degree$ & 25 electrons & 25 electrons & 25 electrons \\
\hline
Optical properties & & &  \\
Focal length & 12.14 mm & 14.67 mm & 43 mm  \\
Entrance pupil diameter & 1.9 mm & 1.25 mm & 2.18 mm  \\
FOV & $26.6 \degree \times 26.6\degree$ & $45 \degree \times 45\degree$ & $16 \degree \times 16\degree$ \\
Depth of field & 0.67 - 2 m & 0.5 m - infinity & 1.5 m - infinity \\ 
Best focus & 1 m & 1 m & 3 m \\
\hline
\end{tabular}
\end{table}

The CCD parameters are similar except for the spectral range. As Pancam mission is to investigate Mars terrain and obtain color images of any information useful to learn more about the Red Planet, that seems logical that the spectrum is wider than the one we choose. The navigation camera provides a 360$\degree$ view of the area where is located the rover. The spectrum was reduced using filters to allow a higher spectral responsivity.
For the optical features, it can be noticed that the parameters have approximately the same size. The depth of field for our camera is restricted but it is due to our aim: stabilizing the camera in front of a rock and carrying out its depth map.
We can conclude that since the MER cameras accomplish well their task, the designed camera which is comparable to them should also be capable of acquiring good images on Mars.

\subsection{Cost and Power Consumption}
The price of a camera depends of its characteristics, especially, of the CCD sensor. The chosen one has large pixels and hence will capture more light than another CCD with half of our pixel size. That means that its performance should be better and thus, the designed camera could be quite expensive. As the exact cost of high quality camera is often not provided, and as finding a similar model to the one planned above is very difficult, it is decided to only determine a price range. Some websites such as ThorLabs have CCD cameras with a resolution of one Megapixel but with half the pixel size desired. As their cost is between 5.000 and 9.000 \euro, we estimate our camera at around 10.000 \euro. Knowing that the Nasa has planned a budget of \$1.5 billion for the new Mars Rover Mission of 2020, this price represents only $0.73\cdot 10^{-3} \ \%$ of the total mission. Taking into consideration that the fee for the artificial source (LED and lens) is negligible compared to the camera cost, it can be concluded that the amount of money needed for buying the designed camera is reasonable.

Regarding the power consumption of our system, it is assumed that the camera could be powered with 2.15 Watts since it is sufficient for supplying the MER Cameras. On the other hand, according to its datasheet (Appendix \ref{LEDdatasheet}), the LED consumption is of 220 mW. On the report of the NASA, a Martian rover needs around 100 Watts and they are provided by solar panels and rechargeable Lithium batteries. The system will then use around 2.4 \% of the rover power. It is tolerable since the camera will not take pictures all the time. It can be argued that the processor consumption should be added. However, the other actions of the robot are also controlled by the CPU, so it is not taken into account here, since it is hard to quantify the power utilized by the algorithms to compute the depth mapping.